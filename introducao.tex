\section{INTRODUÇÃO}

% --- Espaço aéreo geral
A Administração Federal de Aviação dos Estados Unidos (FAA) está evoluindo para a próxima geração de sistemas de transporte aéreo com uma atualização completa de seus sistemas e tecnologias visando reduzir os atrasos, economizar combustível, diminuir a emissão de carbono e aumentar a segurança dos voos. Essa iniciativa integra novas e existentes tecnologias, incluindo a navegação por satélite e comunicações digitais avançadas \cite{faa2015}.

Essa necessidade de ampla atualização surge por ainda se operar mundialmente com vários sistemas e técnicas de monitoramento aéreo ultrapassados. As atuais tecnologias para evitar colisões entre aeronaves, estão se tornando inadequadas. O sistema mais comum para esse fim, Traffic Alert and Collision Avoidance System (TCAS), já é bastante antigo e não é capaz de acompanhar as métricas previstas para a nova geração de sistemas. É necessário, como sujere \citeonline{Chamlou2008}, que algumas tecnologias sejam revistas e novas soluções sejam adotadas.

O sistema TCAS funciona dentro das aeronaves, alertando os pilotos sobre possíveis conflitos com as aeronaves vizinhas. O TCAS I, primeira geração da tecnologia anunciada em 1981, monitora o tráfego ao redor da aeronave em um raio de 65 km e oferece informações de direção e altitude de outras aeronaves. Esta versão oferece ao piloto alertas de colisão na forma Traffic Advisory (TA), que emite um alerta sonoro sobre a proximidade de outra aeronave. Cabe ao piloto a resolução do conflito. O TCAS II foi introduzido em 1989 e é usado na maioria dos equipamentos da aviação comercial atual. O sistema opera de forma sincronizada entre as aeronaves, gerando alertas do tipo Resolution Advisory (RA) fornecendo sugestões de mudança de rota aos pilotos para evitar colisões. O sistema TCAS III foi concebido como uma extensão do sistema TCAS II, permitindo aos pilotos manobras horizontais, além das verticais presentes nas outras versões. Porém, a antena direcional utilizada no posicionamento vertical da aeronave não era precisa o suficiente, e o TCAS IV substituiu o TCAS III em meados de 1990, utilizando informações adicionais de posicionamento global para gerar uma resolução horizontal mais precisa. O desenvolvimento TCAS IV continuou por alguns anos, mas foi abandonado com o surgimento de novas tecnologias, tais como a Automatic Dependent Surveillance-Broadcast (ADS-B), uma tecnologia moderna e acessível desenvolvida para auxiliar o atual sistema baseado em radares no monitoramento de aeronaves civis \cite{Williamson1989}.

% --- Problemática colisão

Colisões entre aeronaves são um grande problema para os sistemas de monitoramento aéreo. Segundo o Escritório de Registros de Acidentes Aéreos\footnote{http://www.baaa-acro.com} (ACRO), no período de 1918 até 2016 foram registrados 23.670 acidentes aéreos com 146.726 mortes, dos quais 7.625 foram causados por falha humana. Isso tende a se agravar com o aumento do tráfego de aeronaves de pequeno porte, Very Light Aircrafts (VLA), pertencentes a empresas pequenas e que operam em aeroportos secundários. Um tráfego aéreo mais denso traz a necessidade de sistemas e algoritmos mais robustos, que possam gerenciar muitas aeronaves simultaneamente.

Os algoritmos computacionais mais comuns para esse propósito, como esclarece \citeonline{Carbone2006}, são formulados como problemas de otimização e a convergência para uma solução não é garantida em um intervalo de tempo finito e determinístico. Assim, ainda segundo \citeonline{Carbone2006}, o algoritmo se comporta de formas diferentes para uma mesma situação de entrada de dados e não é possível prever seu tempo de execução, comportamento exigido por um sistema de controle aéreo em tempo real.

% --- Problemática no contexto do espaço aéreo
No Brasil a problemática é semelhante. Sistemas antigos ainda operam no país, embora já existam alternativas mais baratas e confiáveis. Grande parte do nosso sistema de monitoramento aéreo ainda é baseado em radares. Assim, a eficiência do monitoramento é sujeita a falhas e a prevenção contra colisão de aeronaves é potencialmente problemática. 

% --- Problemática no contexto do Radar Livre
    % --- Solução ADS-B
Para auxiliar o sistema aéreo brasileiro a acompanhar os avanços tecnológicos da FAA, um grupo de pesquisa da Universidade Federal do Ceará (UFC) em Quixadá está desenvolvendo o Sistema de Monitoramento do Espaço Aéreo Radar Livre, uma solução mista de hardware e software baseada na tecnologia Automatic Dependent Surveillance-Broadcast (ADS-B) que proporciona uma base de dados centralizada com todos os dados coletados e uma interface de monitoramento onde as informações podem ser visualizadas graficamente. O sistema opera fora das aeronaves, em um servidor Web, sendo mais adequado ao uso de controladores de voo, e já conta com alguns de seus módulos em funcionamento, como a coleta de dados de aeronaves, um banco de dados online com as informações coletadas e um site, onde as mesmas podem ser acessadas\footnote{Ressalta-se que o sistema possui coletores em funcionamento apenas no estado do Ceará, os quais capturam dados de uma pequena quantidade de aeronaves diariamente. O site do sistema Radar Livre disponibiliza publicamente todas as informações coletadas, e está disponível em www.radarlivre.com}. Dentre suas atuais propostas, destaca-se o desenvolvimento de um módulo de previsão de colisão entre aeronaves e entre aeronaves e acidentes geográficos. 

% --- Explicação do motivo do trabalho nos contextos anteriores

    % --- O trabalho nasce do projeto Radar Livre
    % --- O que o trabalho promete
    % --- Resultados "pretendidos" 

Em aeroportos é comum a utilização de aves de rapina na prevenção contra conflitos entre aves e aeronaves. Por seu objetivo igualmente nobre, o módulo de detecção de colisão do sistema Radar Livre recebeu o apelido de CAPLAN, contração de \textit{Caracara Plancus}, nome científico de uma ave de rapina comum no estado do Ceará. Além de um algoritmo de detecção de colisão eficaz e eficiente, o CAPLAN necessita de alternativas para lidar com milhares de aeronaves simultaneamente. Com foco apenas na detecção de colisões entre aeronaves, o presente trabalho nasce com o intuito de propor e avaliar o desempenho de uma solução para aplicação em grande escala de um algoritmo de detecção de colisão entre aeronaves no espaço aéreo, o qual fará parte do futuro sistema de previsão de colisão do sistema de monitoramento Radar Livre.
